\documentclass[12pt,class=report,crop=false]{standalone}
\usepackage[screen]{../python}

\pagestyle{empty}

\begin{document}

% Commande spécifique
\newcommand{\badletter}[1]{\underline{\textcolor{red}{#1}}}



%====================================================================
\chapitre{Calculatrice polonaise -- Piles}
%====================================================================


  Une \defi{pile} est une suite de données munie de trois opérations de base :
  \begin{itemize}
    \item \defi{empiler}\index{empiler} : on ajoute un élément au sommet de la pile,
    \item \defi{dépiler}\index{depiler@dépiler} : on lit la valeur de l'élément au sommet de la pile et on retire cet élément de la pile,
    \item et enfin, on peut tester si la pile est vide.
  \end{itemize}  

\myfigure{0.6}{
  \tikzinput{fig-piles1}
} 



\newpage

\myfigure{0.5}{
  \tikzinput{fig-piles2}
} 


\newpage


\myfigure{0.6}{
  \tikzinput{fig-piles3}
} 

\newpage

 \centerline{Variable globale \ci{pile}}
 
 \bigskip
 

  \begin{fonction}[\ci{empile()}]
  Usage : \ci{empile(element)} \\
  Entrée : un entier, une chaîne\ldots \\
  Sortie : rien \\
  Action : la pile contient un élément en plus
  
  \medskip
    
  Exemple : si au départ \ci{pile = [5,1,3]} alors, après l'instruction \ci{empile(8)}, la pile vaut \ci{[5,1,3,8]} et si on continue avec l'instruction
\ci{empile(6)}, la pile vaut maintenant \ci{[5,1,3,8,6]}.     
  \end{fonction}


\newpage


  \begin{fonction}[\ci{depile()}]
  Usage : \ci{depile()} \\
  Entrée : rien \\
  Sortie : l'élément du sommet de la pile \\
  Action : la pile contient un élément de moins
  
  \medskip
    
  Exemple : si au départ \ci{pile = [13,4,9]} alors l'instruction \ci{depile()} renvoie la valeur \ci{9} et la pile vaut maintenant \ci{[13,4]} ; si on exécute une nouvelle instruction \ci{depile()}, elle renvoie cette fois la valeur \ci{4} et la pile vaut maintenant \ci{[13]}.
  \end{fonction}

\newpage
  
  \begin{fonction}[\ci{pile_est_vide()}]
  Usage : \ci{pile_est_vide()} \\
  Entrée : rien \\
  Sortie : vrai ou faux \\
  Action : ne fait rien sur la pile
  
  \medskip
    
  Exemple : 
  \begin{itemize}
    \item si \ci{pile = [13,4,9]} alors l'instruction \ci{pile_est_vide()} renvoie \ci{False},
    \item si \ci{pile = []} alors l'instruction \ci{pile_est_vide()} renvoie \ci{True}.
  \end{itemize}
  \end{fonction}
  
  
\newpage

\section*{Calculatrice polonaise}

\begin{itemize}

  \item 
  classique :  $7 + 6$ \quad ; \quad
  polonaise : $7 \ \ 6 \ \ +$

  \item 
  classique : $(10 + 5) \times 3$ \quad ; \quad
  polonaise : $10 \ \ 5 \ \ + \ \ 3 \ \ \times$
  
  \item 
  classique : $10 + 2 \times 3$ \quad ; \quad
  polonaise : $10 \ \ 2 \ \ 3 \ \ \times \ \ +$  
   
  \item 
  classique : $(2 + 8) \times (6 + 11)$ \quad ; \quad
  polonaise : $2 \ \ 8 \ \ + \ \ 6 \ \ 11 \ \ + \ \ \times$ 
\end{itemize}
  
\newpage

\begin{itemize}
  \item On lit l'expression de gauche à droite :
$$\color{red}\underrightarrow{\color{black}2 \ \ 8 \ \ + \ \ 6 \ \ 11 \ \ + \ \ \times}$$ 

  \item Lorsque l'on rencontre un premier opérateur ($+$, $\times$,\ldots) on calcule l'opération \emph{avec les deux membres juste avant cet opérateur} : 
$$\underbrace{\color{red}2 \ \ 8 \ \ +}_{2 + 8} \ \ 6 \ \ 11 \ \ + \ \ \times$$ 
   
  \item On remplace cette opération par le résultat :
$$\underbrace{\color{red} 10}_{\substack{\text{résultat de }\\2 + 8}} \ \ 6 \ \ 11 \ \ + \ \ \times$$
  
  \item On continue la lecture de l'expression (on cherche le premier opérateur et les deux termes juste avant) :
 $$10 \ \ \underbrace{\color{red}6 \ \ 11 \ \ +}_{6 + 11 = 17}  \ \ \times
 \qquad \text{ devient } \qquad 
10 \ \ 17  \ \ \times  \qquad \text{ qui vaut } \qquad 170$$
  
  \item À la fin il ne reste qu'une valeur, c'est le résultat ! (Ici $170$.)
\end{itemize}

 
\newpage

\begin{itemize}
  \item $8 \ \ 2 \ \ \div \ \ 3 \ \ \times \ \ 7 \ \ +$  
  
   
 \bigskip
 
$$\underbrace{8 \ \ 2 \ \ \div}_{8 \div 2 = 4} \ \ 3 \ \ \times \ \ 7 \ \ +
\quad\text{ devient }\quad
\underbrace{4 \ \ 3 \ \ \times}_{4 \times 3 = 12} \ \ 7 \ \ + 
\quad\text{ devient }\quad
 12 \ \ 7 \ \ +
\quad\text{ qui vaut }\quad
19$$ 
 
 
 \bigskip
 
 \bigskip
 
  \item $11 \ \ 9 \ \ 4 \ \ 3 \ \ + \ \ - \ \ \times$  
   
 \bigskip
  
$$11 \ \ 9 \ \ \underbrace{4 \ \ 3 \ \ +}_{4+3=7} \ \ - \ \ \times
\quad\text{ devient }\quad
11 \ \ \underbrace{9 \ \ 7 \ \ -}_{9 - 7 = 2} \ \ \times 
\quad\text{ devient }\quad
11 \ \ 2 \ \ \times
\quad\text{ qui vaut }\quad
22$$ 
\end{itemize}  
 
\newpage

 \begin{algorithme}
  \sauteligne 
 \begin{itemize}
   \item
   \begin{itemize}
     \item Entrée : une expression en écriture polonaise (une chaîne de caractères).
     \item Sortie : la valeur de cette expression.
     \item Exemple : \ci{"2 3 + 4 *"} (le calcul $(2+3) \times 4$) donne $20$.        
   \end{itemize}

  \item Partir avec une pile vide.   
   
   \item Pour chaque élément de l'expression (lue de gauche à droite) :
   \begin{itemize}
     \item si l'élément est un nombre, alors ajouter ce nombre à la pile,
     
     \item si l'élément est une opération, alors :
       \begin{itemize}
         \item dépiler une fois pour obtenir un nombre $b$,
         \item dépiler une seconde fois pour obtenir un nombre $a$,
         \item calculer $a+b$ ou $a \times b$ selon l'opération,
         \item ajouter ce résultat à la pile.
       \end{itemize}
     \end{itemize}   
   \item À la fin, la pile ne contient qu'un seul élément, c'est le résultat du calcul.
   
 \end{itemize}  
 \end{algorithme}

 
\newpage

\begin{fonction}[\ci{calculatrice_polonaise()}]
  Usage : \ci{calculatrice_polonaise(expression)} \\
  Entrée : une expression en notation polonaise (chaîne de caractères) \\
  Sortie : le résultat du calcul \\
  Action : utilise une pile
   
  \medskip
   
  Exemple : 
  \begin{itemize}
    \item \ci{calculatrice_polonaise("2 3 4 + +")} renvoie \ci{9}
    \item \ci{calculatrice_polonaise("2 3 + 5 *")} renvoie \ci{25}
    \end{itemize}    
\end{fonction}  
  

\end{document}
