\documentclass[12pt,class=report,crop=false]{standalone}
\usepackage[screen]{../python}

\pagestyle{empty}

\begin{document}

% Commande spécifique
\newcommand{\badletter}[1]{\underline{\textcolor{red}{#1}}}



%====================================================================
\chapitre{Listes II}
%====================================================================




\section*{Trancher des listes}
  
  \begin{itemize}
    \item Tu connais déjà \ci{maliste[a:b]} qui renvoie la sous-liste des éléments du rang $a$ au rang $b-1$.
    
    \item \ci{maliste[a:]} renvoie la liste des éléments du rang $a$ jusqu'à la fin.
      
    \item \ci{maliste[:b]} renvoie la liste des éléments du début jusqu'au rang $b-1$.
    
    \item \ci{maliste[-1]} renvoie le dernier élément, \ci{maliste[-2]} renvoie l'avant-dernier élément,\ldots
    \end{itemize}
        
        
 
\bigskip       
          
\textbf{Exercice.} 
  \myfigure{0.7}{
  \tikzinput{fig-listes-4}
}  
  Avec \ci{maliste = [7,2,4,5,3,10,9,8,3]}, 
que renvoient les instructions suivantes ?


\begin{itemize}
  \item \ci{maliste[3:5]}
  \item \ci{maliste[4:]}
  \item \ci{maliste[:6]}
  \item \ci{maliste[-1]}
\end{itemize}

\newpage

\section*{Trouver le rang d'un élément} 

   \ci{liste.index(element)} renvoie la première position à laquelle l'élément a été trouvé. Exemple : avec \ci{liste = [12, 30, 5, 9, 5, 21]},
   \ci{liste.index(5)} renvoie $2$.


\bigskip
\bigskip

  Si on souhaite juste savoir si un élément appartient à une liste, alors l'instruction :\\
  \centerline{\ci{element in liste}}  
  renvoie \ci{True} ou \ci{False}.
  Exemple : avec \ci{liste = [12, 30, 5, 9, 5, 21]},
   \og\ci{9 in liste}\fg{} est vrai, alors que \og\ci{8 in liste}\fg{} est faux.


\newpage

\section*{Liste par compréhension}
  
 $$E = \{0,2,4,6,8,10\}$$
 
  $$E = \{ x \in \Nn \mid x \le 10 \text{ et } x \text{ est pair} \}$$
  
\bigskip
\bigskip  
  
  Liste par compréhension \Python{}
  \bigskip  
  
  \begin{itemize}
    \item Partons d'une liste, par exemple \ci{maliste = [1,2,3,4,5,6,7,6,5,4,3,2,1]}.
 \bigskip     
    \item La commande \ci{liste_doubles = [ 2*x for x in maliste ]} renvoie une liste qui contient les doubles des éléments de la liste \ci{maliste}. C'est donc la liste 
    \ci{[2,4,6,8,...]}.
\bigskip      
    \item La commande \ci{liste_carres = [ x**2 for x in maliste ]} renvoie la liste des carrés des éléments de la liste initiale. C'est donc la liste \ci{[1,4,9,16,...]}.
\bigskip      
    \item La commande \ci{liste_partielle = [x for x in maliste if x > 2]}
    extrait la liste composée des seuls éléments strictement supérieurs à $2$. C'est donc la liste \ci{[3,4,5,6,7,6,5,4,3]}.
	\end{itemize}
	
	
\newpage

\section*{Liste de listes}
  

  
  Une liste peut contenir d'autres listes, par exemple :\\
  \centerline{\ci{maliste = [ ["Harry", "Hermione", "Ron"], [101,103] ]}}
   contient deux listes. 
   
\bigskip
\bigskip 
   
  Listes qui contiennent des listes d'entiers : des \defi{tableaux}. 
  
  Par exemple : \\
  \centerline{\ci{tableau = [ [2,14,5], [3,5,7], [15,19,4], [8,6,5] ]}}

  $$\begin{array}{ccc}2&14&5\\3&5&7\\15&19&4\\8&6&5\end{array}$$
  
\bigskip 
  
  Alors \ci{tableau[i]} renvoie la sous-liste de rang $i$, alors que
  \ci{tableau[i][j]} renvoie l'entier situé au rang $j$ dans la sous-liste de rang $i$. Par exemple :
  \begin{itemize}
  \item \ci{tableau[0]} renvoie la liste \ci{[2,14,5]},
  \item \ci{tableau[1]} renvoie la liste \ci{[3,5,7]},
  \item \ci{tableau[0][0]} renvoie l'entier \ci{2},
  \item \ci{tableau[0][1]} renvoie l'entier \ci{14},
  \item \ci{tableau[2][1]} renvoie l'entier \ci{19}.
\end{itemize}

\end{document}
