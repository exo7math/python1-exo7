\documentclass[11pt,class=report,crop=false]{standalone}
\usepackage[screen]{../python}

\pagestyle{empty}

\begin{document}


%====================================================================
\chapitre{Si ... alors ...}
%====================================================================



\mybox{
\myfigure{0.7}{
  \tikzinput{fig-sialors-cours-1}
} }

\newpage
\section*{Si ... alors ... sinon ...}

\mybox{
\myfigure{0.7}{
  \tikzinput{fig-sialors-cours-2}
} }


\newpage


\section*{Entrée au clavier}


\begin{itemize}
  \item \ci{input()} met en pause l'exécution du programme et attend de l'utilisateur un texte
  
  \bigskip
  
  \item Cette commande renvoie une chaîne de caractères.
   
  \bigskip
   
  \item \ci{int}. Si on veut un entier, il faut convertir la chaîne. 
  
  
  Exemple. Si \ci{age_chaine} vaut  \ci{"17"}, alors \ci{int(age_chaine)} vaut l'entier \ci{17}. 


  \bigskip
   
  \item \ci{float}. Si on veut un nombre flottant, il faut convertir la chaîne. 
  
  
  Exemple. Si \ci{pi_chaine} vaut  \ci{"3.14"}, alors \ci{float(pi_chaine)} vaut le nombre flottant \ci{3.14}. 
     
  \bigskip
   
  \item \ci{str} convertit un nombre en une chaîne. 
  
  Exemple.  \ci{str(17)} renvoie la chaîne \ci{"17"} ; si \ci{age = 17}, alors \ci{str(age)} renvoie également \ci{"17"}.
\end{itemize}


\newpage

%%%%%%%%%%%%%%%%%%%%%%%%%%%%%%%%%%%%%%%%%%%%%%%%%%%%%%%%%%%%%%%%
%%%%%%%%%%%%%%%%%%%%%%%%%%%%%%%%%%%%%%%%%%%%%%%%%%%%%%%%%%%%%%%%

\section*{Le module \og{}random\fg{}}

Le module \ci{random} génère des nombres comme s'ils étaient tirés au hasard.

   
  \bigskip
  
\begin{itemize}
  \item En début du programme :\\
  \centerline{\ci{from random import *}}
    
  \bigskip 
  \item \ci{randint(a,b)}\index{randint@\ci{randint}} renvoie un entier au hasard compris entre $a$ et $b$.
  
  Exemple. \ci{n = randint(1,6)}, $n$ est un entier tiré au hasard avec $1 \le n \le 6$.
  Si on recommence l'instruction \ci{n = randint(1,6)}, $n$ prend une nouvelle valeur. C'est comme si on effectuait le lancer d'un dé à $6$ faces.
     
     
  \bigskip
  \item \ci{random()} renvoie un nombre flottant compris entre $0$ et $1$.  
  
  Exemple. Avec \ci{x = random()}, alors $x$ est un nombre flottant avec $0 \le x < 1$.
\end{itemize}



\newpage

\section*{Booléens}


\sauteligne
\begin{itemize}
  \item Un \defi{booléen}\index{booleen@booléen} est une donnée qui vaut soit la valeur \og{}vrai\fg{}, soit la valeur \og{}faux\fg{}. En \Python{} les valeurs sont \ci{True}\index{true@\ci{True}} et \ci{False}\index{false@\ci{False}} (avec une majuscule).
     
  \bigskip
    
  \item On obtient un booléen par exemple comme résultat de la comparaison de deux nombres.
  Par exemple \ci{7 < 4} vaut \ci{False} (car $7$ n'est pas plus petit que $4$). 
  Vérifie que \ci{print(7 < 4)} affiche \ci{False}.
    
\end{itemize} 
     
  \bigskip
  \bigskip  
  
    Voici les principales comparaisons :
  \begin{itemize}
    \item \textbf{Test d'égalité :}\quad \ci{a == b}\index{\ci{==}}
    	\item \textbf{Test inférieur strict :}\quad \ci{a < b}
    	\item \textbf{Test inférieur large :}\quad \ci{a <= b}\index{\ci{<=}}
    	\item \textbf{Test supérieur :}\quad \ci{a > b} \quad ou \quad \ci{a >= b}\index{\ci{>=}}
    	\item \textbf{Test non égalité :}\quad \ci{a != b}\index{\ci{"!}\ci{=}}
  \end{itemize}
   
   Par exemple \ci{6*7 == 42} vaut \ci{True}.

  
\newpage

 \section*{Booléens}


   
   \mybox{
   \textbf{ATTENTION !}   
   L'erreur classique est de confondre \og{}\ci{a = b}\fg{} et \og{}\ci{a == b}\fg{}.
   \begin{itemize}
     \item \textbf{Affectation.} \ci{a = b}
     met le contenu de la variable \ci{b} dans la variable \ci{a}.
     \item \textbf{Test d'égalité.} \ci{a == b} teste si les contenus de \ci{a} et de \ci{b} sont égaux et vaut \ci{True} ou \ci{False}.
   \end{itemize}
   }
   
   
\newpage

 \section*{Booléens}
   
 \begin{itemize} 
      
  \item On peut comparer autre chose que des nombres. Par exemple 
  \og{}\ci{car == "A"}\fg{} teste si la variable \ci{car} vaut \ci{"A"} ; \og{}\ci{il_pleut == True}\fg{} teste si la variable \ci{il_pleut} est vraie\ldots
  
   \bigskip  
    
  \item Les booléens sont utiles dans le test \og{}si \ldots{} alors \ldots\fg{} et dans les boucles \og{}tant que \ldots{} alors \ldots\fg{}.
  
    \bigskip  
    
  \item \textbf{Opérations entre les booléens.}
  \index{operation logique@opération logique}
  Si $P$ et $Q$ sont deux booléens, on peut définir de nouveaux booléens.
  \begin{itemize}
    \item \textbf{Et logique.}\index{et}\quad \og{}\ci{P and Q}\fg{}\index{and@\ci{and}} est vrai si et seulement si $P$ et $Q$ sont vrais.
    	\item \textbf{Ou logique.}\index{ou}\quad \og{}\ci{P or Q}\fg{}\index{or@\ci{or}} est vrai si et seulement si $P$ ou $Q$ est vrai.
    	\item \textbf{Négation.}\index{non}\quad \og{}\ci{not P}\fg{}\index{not@\ci{not}} est vrai si et seulement si $P$ est faux.
  \end{itemize}  
   
    \bigskip  
     
  Exemple. \og{}\ci{(2+2 == 2*2) and (5 < 3)}\fg{} renvoie \ci{False}, car
  même si on a bien $2+2 = 2 \times 2$, l'autre condition n'est pas remplie car $5 < 3$ est faux.
  
  
\end{itemize}

\end{document}
